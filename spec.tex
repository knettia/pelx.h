\documentclass{article}
\usepackage{amsmath, amssymb, amsthm}
\usepackage{mathtools}
\usepackage{graphicx}
\usepackage{listings}
\usepackage{array}
\usepackage[margin=1in]{geometry}

\usepackage[colorspec=0.9]{draftwatermark}
\SetWatermarkText{DRAFT}
\SetWatermarkScale{7.5}

\title{Specification of the PELX 0.1.0 File Format}

\author{A. C. Gäßler}

\date{2025}

\begin{document}

\maketitle

\section*{Abstract}

The PELX format is an image encoding scheme which relies on indexed colour palettes and true colour data to display a countless amount of colour combinations after the file was initially stored.

\section{Notation}

Let:
\begin{itemize}
	\item $\mathbb{B} = \{0, 1\}$ denote the set of binary values.
	\item $\mathbb{N}_k$ denote the set of unsigned integers representable in $k$ bits.
	\item $\mathbb{B}_n$ denote a sequence of $n$ bits.
	\item $[\![a, b]\!]$ denote the set of integers from $a$ to $b$, inclusive.
\end{itemize}

\section{PELX Header}

The header $\mathcal{H} \in \mathbb{B}_{32 \times 8}$ consists of the following fields:

\begin{align*}
	\mathcal{H}_{\text{magic}} &\in (\mathbb{N}_8)5, &\text{must equal ASCII (``PELX\textbackslash0'')} \\
	\mathcal{H}_{\text{header\_size}} &\in \mathbb{N}_{32} \\
	\mathcal{H}_{\text{palette\_offset}} &\in \mathbb{N}_{32} \\
	\mathcal{H}_{\text{width}}, \mathcal{H}_{\text{height}} &\in \mathbb{N}_{16}, &\text{must be } > 0 \\
	\mathcal{H}_{\text{palette\_channel\_count}}, \mathcal{H}_{\text{true\_channel\_count}} &\in \{3, 4\} \\
	\mathcal{H}_{\text{palette\_count}} &\in \mathbb{N}_{16}, &\text{must be } > 0 \\
	\mathcal{H}_{\text{reserved}} &\in (\mathbb{N}_8)5
\end{align*}

\section{File Layout}

Let $\mathcal{F}$ denote a PELX file. It is partitioned into:
\[
	\mathcal{F} = \underbrace{\mathcal{H}}_{\text{Header}} \quad \| \quad
	\underbrace{\mathcal{P}}_{\text{Palette Definitions}} \quad \| \quad
	\underbrace{\mathcal{D}}_{\text{Pixel Data}}
\]

where:
\begin{itemize}
	\item $\|\;$ denotes concatenation.
	\item $\mathcal{P}$ begins at byte offset $\mathcal{H}_{\text{palette\_offset}}$.
	\item $\mathcal{D}$ begins at byte offset $\mathcal{H}_{\text{header\_size}}$.
\end{itemize}

\section{Palette Entry}

Each palette entry $p \in \mathcal{P}$ is defined as:
\[
	p = (r, g, b, a) \in (\mathbb{N}_8)c \quad \text{where } c = \mathcal{H}_{\text{palette\_channel\_count}} \in \{3, 4\}
\]

There are $\mathcal{H}_{\text{palette\_count}}$ such entries.

\section{Pixel Data Encoding}

Let $T$ be $\mathcal{H}_{\text{true\_channel\_count}} \in \{3, 4\}$

Each pixel datum in $\mathcal{D}$ is of the form:
\[
	d \in \texttt{TaggedUnion} \in \{ \text{Void}, \text{True}, \text{Pale} \}
\]

With:
\begin{align*}
	\text{Void} &::= 0\text{x}00 \\
	\text{True} &::= 0\text{x}01 \| (\mathbb{N}_8)T \\
	\text{Pale} &::= 0\text{x}02 \| \mathbb{N}_8
\end{align*}

The semantics are:
\begin{itemize}
	\item Void: represents a fully transparent or black pixel (R=G=B=A=0).
	\item True: followed by $T$ colour channels (RGB[A]) encoded inline.
	\item Pale: followed by an index $i$ into the palette, with $i \in [\![0, \mathcal{H}_{\text{palette\_count}} - 1]\!]$.
\end{itemize}

\section{Validity}

A valid PELX file must satisfy:

\begin{itemize}
	\item $\mathcal{H}_{\text{magic}} = \text{(`P', `E', `L', `X', `\textbackslash0')}$.
	\item $\mathcal{H}_{\text{width}} > 0 \land \mathcal{H}_{\text{height}} > 0$.
	\item $\mathcal{H}_{\text{true\_channel\_count}} \in \{3, 4\}$.
	\item $\mathcal{H}_{\text{palette\_channel\_count}} \in \{3, 4\}$.
	\item $\mathcal{H}_{\text{palette\_count}} > 0$.
	\item All pixel data in $\mathcal{D}$ must decode to exactly $\mathcal{H}_{\text{width}} \times \mathcal{H}_{\text{height}}$ pixels.
\end{itemize}

\section{Summary}

\begin{center}
	\renewcommand{\arraystretch}{1.2}

	\begin{tabular}{|c|l|}
		\hline
			Tag & Meaning \\
		\hline
			0x00 & Void pixel \\
			0x01 & True colour pixel (inline RGB[A]) \\
			0x02 & Palette pixel (index into palette block) \\
		\hline
	\end{tabular}

\end{center}

\section{Notes}

Currently, the palette block $\mathcal{P}$ is not used, but reserved for future versions where it may be augmented to include named palettes or metadata fields.

\end{document}
